% !TEX encoding = UTF-8 Unicode

\thispagestyle{empty}                   % no headers and footers
\lithuanian

\begin{center}
	\vspace*{5mm}	
	
	VILNIAUS UNIVERSITETAS \\

	
	
	\vspace{45mm}
	
	Indrė Žliobaitė
	
	\vspace{15mm}
	
  ADAPTYVUS MOKYMO IMTIES FORMAVIMAS

  \vspace{30mm}
  
  Daktaro disertacijos santrauka\\[-6pt]  
  Fiziniai mokslai, informatika (09P)
  
  \vspace{60mm}
  
  Vilnius, 2010
  
\end{center}

\newpage
\thispagestyle{empty}                   % no headers and footers
\lithuanian

\begin{singlespace}
\noindent Disertacija rengta 2006 - 2010 metais Vilniaus universitete bendradarbiaujant su Bangoro universiteto (Didžioji Britanija) ir Eindhoveno technologijų universiteto (Nyderlandai) mokslininkais.

\vspace{1cm}

\noindent Mokslinis vadovas:
\vspace{0.5cm}

\begin{singlespace}
prof. habil. dr. Šarūnas Raudys (Vilniaus universitetas, fiziniai mokslai, informatika - 09P).
\end{singlespace}

\vspace{1cm}
	
\noindent\textbf{Disertacija ginama Vilniaus universiteto Informatikos mokslo krypties taryboje:}

\vspace{0.5cm}
	
\noindent Pirmininkas

prof. dr. Algimantas Juozapavičius (Vilniaus universitetas, fiziniai mokslai, informatika - 09P),

\noindent Nariai:

doc. dr. Algirdas Bastys (Vilniaus universitetas, fiziniai mokslai, informatika - 09P),

prof. habil. dr. Henrikas Pranevičius (Kauno technologijos universitetas, fiziniai mokslai, informatika - 09P),

prof. habil. dr. Rimvydas Simutis (Kauno technologijos universitetas, technologijos mokslai, informatikos inžinerija - 07T), 

dr. Julius Žilinskas (Matematikos ir informatikos institutas, fiziniai mokslai, informatika - 09P).

\vspace{1cm}

\noindent Oponentai:

doc. dr. Minija Tamošiūnaitė (Vytauto Didžiojo universitetas, fiziniai mokslai, informatika - 09P),

doc. dr. Pranas Vaitkus (Vilniaus universitetas, fiziniai mokslai, matematika - 01P).
\\


\vspace{1cm}
	
\noindent Disertacija bus ginama viešame Informatikos mokslo krypties tarybos posėdyje 2010 m. balandžio mėn. 1 d. 11 val. 

Adresas: Vilniaus universiteto Matematikos ir informatikos fakulteto Nuotolinių studijų centras, Šaltinių 1A, LT-03225 Vilnius.

\vspace{1cm}
\noindent Disertacijos santrauka išsiuntinėta 2010 m. kovo mėn. 1 d.

\noindent Disertaciją galima peržiūrėti Vilniaus universiteto bibliotekoje.
\end{singlespace}
	
%\cleardoublepage
\newpage

\thispagestyle{empty}                   % no headers and footers
\lithuanian

\begin{center}
	\vspace*{5mm}	
	
	VILNIUS UNIVERSITY \\

	
	
	\vspace{45mm}
	
	Indrė Žliobaitė
	
	\vspace{15mm}
	
  ADAPTIVE TRAINING SET FORMATION

  \vspace{30mm}
  
  Summary of doctoral dissertation\\[-6pt]  
  Physical sciences, informatics (09P)
  
  \vspace{60mm}
  
  Vilnius, 2010
  
\end{center}

\newpage
\thispagestyle{empty}                   % no headers and footers
\lithuanian

\begin{singlespace}
\noindent The dissertation work was carried out at Vilnius University from 2006 to 2010 in cooperation with Bangor University (UK) and Eindhoven University of Technology (the Netherlands) researchers.

\vspace{1cm}

\noindent Scientific supervisor:
\vspace{0.5cm}
\begin{singlespace}
prof. habil. dr. Šarūnas Raudys (Vilnius University, physical sciences, informatics - 09P).
\end{singlespace}
\vspace{1cm}
	
\noindent\textbf{The defense council:}

\vspace{0.5cm}
	
\noindent Chairman

prof. dr. Algimantas Juozapavičius (Vilnius University, physical sciences, informatics - 09P),

\noindent Members:

doc. dr. Algirdas Bastys (Vilnius University, physical sciences, informatics - 09P),

prof. habil. dr. Henrikas Pranevičius (Kaunas University of Technology, physical sciences, informatics - 09P),

prof. habil. dr. Rimvydas Simutis (Kaunas University of Technology, technological sciences, informatics engineering - 07T), 

dr. Julius Žilinskas (Institute of Mathematics and Informatics, physical sciences, informatics - 09P).

\vspace{1cm}

\noindent Opponents:

doc. dr. Minija Tamošiūnaitė (Vytautas Magnus University, physical sciences, informatics - 09P),

doc. dr. Pranas Vaitkus (Vilnius university, physical sciences, mathematics - 01P).
\\


\vspace{1cm}
	
\noindent The dissertation will be defended at the public meeting of the council on the 1st of April, 2010 at 11:00. \\
Address: VU MIF Distance Learning Center, Šaltinių 1A, LT-03225 Vilnius.

\vspace{1cm}
\noindent The summary of the dissertation was distributed on the 1st of March, 2010.

\noindent The dissertation is available at the library of Vilnius University.
\end{singlespace}